% ------------------------------------------------------------------------------
% Este fichero es parte de la plantilla LaTeX para la realización de Proyectos
% Final de Grado, protegido bajo los términos de la licencia GFDL.
% Para más información, la licencia completa viene incluida en el
% fichero fdl-1.3.tex

% Copyright (C) 2012 SPI-FM. Universidad de Cádiz
% ------------------------------------------------------------------------------

A continuación se recogen las instrucciones necesarias para evolucionar el software. Este manual está dirigido a los desarrolladores que pretenden extender o modificar el código fuente, con el fin de incorporar nuevas funcionalidades o modificar las ya existentes. A lo largo de este capítulo se deberán hacer referencias explicitas a aquellos epígrafes de los capítulos de Diseño, Construcción y Pruebas del Sistema que resulten de interés.

\section{Introducción}
Resumen de los principales objetivos, ámbito, características y alcance del software desarrollado.

\section{Preparación del entorno de trabajo}
Descripción de los requisitos (hardware y software) previos. Datos de interés relativos al control de versiones del software. Detalles sobre la instalación en local del entorno de desarrollo y, si fuesen necesarios, de otros componentes como bases de datos, servidores de aplicaciones, etc. 

\section{Consideraciones generales sobre el desarrollo}
Aspectos importantes a tener en cuenta a la hora de modificar y extender el código fuente, guías de estilo, etc. Asimismo, se detallarán las directrices que sean de aplicación a la hora de realizar pruebas sobre las nuevas mejoras introducidas.

\section{Instrucciones para construcción y despliegue}
Secuencia de pasos requeridos para llevar a cabo la compilación del código fuente y así poder construir y depurar el software sobre una máquina de desarrollo.
